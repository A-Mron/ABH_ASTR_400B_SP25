% mnras_template.tex 
%
% LaTeX template for creating an MNRAS paper
%
% v3.3 released April 2024
% (version numbers match those of mnras.cls)
%
% Copyright (C) Royal Astronomical Society 2015
% Authors:
% Keith T. Smith (Royal Astronomical Society)

% Change log
%
% v3.3 April 2024
%   Updated \pubyear to print the current year automatically
% v3.2 July 2023
%	Updated guidance on use of amssymb package
% v3.0 May 2015
%    Renamed to match the new package name
%    Version number matches mnras.cls
%    A few minor tweaks to wording
% v1.0 September 2013
%    Beta testing only - never publicly released
%    First version: a simple (ish) template for creating an MNRAS paper

%%%%%%%%%%%%%%%%%%%%%%%%%%%%%%%%%%%%%%%%%%%%%%%%%%
% Basic setup. Most papers should leave these options alone.
\documentclass[fleqn,usenatbib]{mnras}

% MNRAS is set in Times font. If you don't have this installed (most LaTeX
% installations will be fine) or prefer the old Computer Modern fonts, comment
% out the following line
\usepackage{newtxtext,newtxmath}
% Depending on your LaTeX fonts installation, you might get better results with one of these:
%\usepackage{mathptmx}
%\usepackage{txfonts}

% Use vector fonts, so it zooms properly in on-screen viewing software
% Don't change these lines unless you know what you are doing
\usepackage[T1]{fontenc}

% Allow "Thomas van Noord" and "Simon de Laguarde" and alike to be sorted by "N" and "L" etc. in the bibliography.
% Write the name in the bibliography as "\VAN{Noord}{Van}{van} Noord, Thomas"
\DeclareRobustCommand{\VAN}[3]{#2}
\let\VANthebibliography\thebibliography
\def\thebibliography{\DeclareRobustCommand{\VAN}[3]{##3}\VANthebibliography}


%%%%% AUTHORS - PLACE YOUR OWN PACKAGES HERE %%%%%

% Only include extra packages if you really need them. Avoid using amssymb if newtxmath is enabled, as these packages can cause conflicts. newtxmatch covers the same math symbols while producing a consistent Times New Roman font. Common packages are:
\usepackage{graphicx}	% Including figure files
\usepackage{amsmath}	% Advanced maths commands

%%%%%%%%%%%%%%%%%%%%%%%%%%%%%%%%%%%%%%%%%%%%%%%%%%

%%%%% AUTHORS - PLACE YOUR OWN COMMANDS HERE %%%%%

% Please keep new commands to a minimum, and use \newcommand not \def to avoid
% overwriting existing commands. Example:
%\newcommand{\pcm}{\,cm$^{-2}$}	% per cm-squared

%%%%%%%%%%%%%%%%%%%%%%%%%%%%%%%%%%%%%%%%%%%%%%%%%%

%%%%%%%%%%%%%%%%%%% TITLE PAGE %%%%%%%%%%%%%%%%%%%

% Title of the paper, and the short title which is used in the headers.
% Keep the title short and informative.
\title[Short title, max. 45 characters]{Kinematic Evolution of Dark Matter Halos After Major Merger}

% The list of authors, and the short list which is used in the headers.
% If you need two or more lines of authors, add an extra line using \newauthor
\author[Aaron B. Hernandez et al.]{
Aaron B. Hernandez,$^{1}$
\\
% List of institutions
$^{1}$ASTR 400B SP25, University Of Arizona, Tucson AZ, USA\\
}

% These dates will be filled out by the publisher
%\date{Accepted XXX. Received YYY; in original form ZZZ}

% Prints the current year, for the copyright statements etc. To achieve a fixed year, replace the expression with a number. 
\pubyear{\the\year{2025}}

% Don't change these lines
\begin{document}
\label{firstpage}
\pagerange{\pageref{firstpage}--\pageref{lastpage}}
\maketitle

% Abstract of the paper
\section{Introduction}
\subsection{The Topic}
The kinetic evolution of the Dark matter halo after a major merger is described by how objects in the halo move and how that movement changes through the merging process. The Dark matter halo is a galaxy region that stretches out far beyond the galaxy. This region is dominated by non-visible dark matter as the name suggests and is crucial to maintaining the galaxies internal structure. In galaxy mergers the halo governs over the orbital dynamics and is what drives the eventual combining into one singular galaxy remanent.  
\subsection{Why it Matters}
The evolution of galaxy kinematics after a merger gives key insights into the newly merged galaxies characteristics. Traits like star formation, mass distribution, and velocity distributions all stem from the kinematics of the galaxy. The Milky Way (MW) and Andromeda (M31) galaxies will eventually merge into one galaxy in around 5.5 billion years and even though we will not be around to witness it the dynamics of this future galaxy are a mystery and mysteries are made to be solved. 

\subsection{Current Understanding}
From modern simulations galactic mergers cause extended mass distributions for the final merger and an increased central density \citep[e.g.][]{Drakos2019bMNRAS} as seen in Fig\ref{fig:fig 1}.  This extension changes the inner kinematics of the newly merged galaxy and as time progresses the kinematics will 'stabilize' and form a new system of movement for stars and new dark matter and stellar mass distributions. This change in kinematic affects gas as well, during the merger gas clouds of both galaxies can merges and create regions of significant star formation otherwise referred to as 'starbursts' \citep[e.g.][]{Ejdetjärn2025}.
\begin{figure}
	\includegraphics[width=\columnwidth]{mass_prof.png}
    \caption{Results of 4 merger simulations form \cite{Drakos2019aMNRAS}. The bottom row of plots are the mass profiles. From the plots we can see that at as expected the final merger (solid black) has a higher density profile ($\rho$)}
    \label{fig:fig 1}
\end{figure}
\subsection{Open Questions}
Although M31 and the MW are the two giants in focus, the Triangulum Galaxy (M33) is a satellite galaxy of M31 and it will be subject to the kinematics of the merger of M31 and MW. This poses the question, will M33 be bound to the galaxy for will it have the velocity to escape the pull of the merger?. Other questions include, 'How does the assembly of a central galaxy affect the mass profile and shape of the dark matter halo?' \cite{Abadi2010MNRAS}, 'How do mergers effect galaxy characteristics like gas kinematics and star formation history?' \cite{Ejdetjärn2025}, and 'How are the structural properties of the dark matter halo related to it's growth history?' \cite{Drakos2019aMNRAS}.

% Select between one and six entries from the list of approved keywords.
% Don't make up new ones.
%%%%%%%%%%%%%%%%%%%%%%%%%%%%%%%%%%%%%%%%%%%%%%%%%%
%All papers should start with an Introduction section, which sets the work
%in context, cites relevant earlier studies in the field by \citet{Fournier1901},
%and describes the problem the authors aim to solve \citep[e.g.][]{vanDijk1902}.
%Multiple citations can be joined in a simple way like \citet{deLaguarde1903, delaGuarde1904}.
%%%%%%%%%%%%%%%%% BODY OF PAPER %%%%%%%%%%%%%%%%%%

\section{Proposal}

\subsection{This Proposal}
\label{sec:This Proposal}
The purpose of this paper will be to answer the questions, \textit{What is the escape speed of the remnant as a function of radius} and \textit{How does the escape velocity vary when computed treating the Dark Matter Halo as a point mass vs fitting a Hernquist profile and using analytic potential. Finally I will compare the two escape velocities and see how they compare and what features make them different. }
\subsection{Methods}
\label{sec:Methods}
By the end of this project I will have obtained two equations for the escape velocity as a function of radius. One being the well known:
\begin{equation}
    v_{esc}(R) = \sqrt{\frac{GM}{R}} \label{EQ1}
\end{equation}
And the other will be calculated using the potential of the at each given radius which will take the form of:
\begin{equation}
    v_{esc}(R) = \sqrt{2|\Phi|} \label{EQ2}
\end{equation}

The potential $(\Phi)$ will consist of three components, one from the bulge, one from the disk, and one from the Halo. The potentials from the Bulge and Disk will be calculated using EQ\ref{EQ1} with their masses being treated as point masses, but the Halo mass will be calculated by fitting a Hernquist profile to the merged dark matter profiles of MW and M31. When the Hernquist profile is fitted it will give the mass (M) and scale length (a) that is needed for the potential equation:
\begin{equation}
    \Phi(R) = \frac{-GM}{(R+a)}
    \label{EQ3}
\end{equation}
The potentials of the halo,disk,and bulge will be summed together and used in EQ\ref{EQ2}.

\begin{figure}
	% To include a figure from a file named example.*
	% Allowable file formats are eps or ps if compiling using latex
	% or pdf, png, jpg if compiling using pdflatex
	\includegraphics[width=\columnwidth]{image.png}
    \caption{Distribution of simulated Dark Matter particles and gas particles in black and red/green respectfully. The distribution shows how the Dark matter in the halo is very spread out and is in no way a 'point mass'. Figure from \cite{Abadi2010MNRAS} }
    \label{fig:fig 2}
\end{figure}
I will combine codes from previous homework and labs with two new codes that will be written just for this project. I will use the Homework 2,3,and 5 codes which are the Python scripts that read data files, sum the total mass of components, and fit a Hernquist profile respectively. I will also use code from Lab 4 for the Hernquist profile. The first new script will be a code that takes the total mass of the merged MW and M31 galaxies and plots $v_{esc}$ vs radius and the other that plots $v_{esc,Hernquist}$ vs radius. Both codes will use all three particle types but the disk and bulge (types 2 and 3) will only be used to calculate the total mass of the disk and bulge. Type 1 particles (Halo) will be the type that has a Hernquist profile fitted to. I will use a snapshot after M31 and the MW merge and a clear remanent is determined. Snapshot 650 from the low resolution will be an ideal choice because it will be long after the merger of the two galaxies and the dark mater halo will have stabilized. Low resolution snapshots are also ideal as they contain the necessary mass values while not containing unnecessary data on individual particles. A stabilized halo is required for the study of its kinematics. The snapshot will be obtained form the simulation by \citep[][]{vanderMarel2012ApJ} of the merger of M31 and the Milky Way. 
I will also draw a comparison between the two and see just how much the point mass assumption varies from a more accurate/thought out approach like method\ref{EQ2}. A third plot of the two escape velocities against each other will show how far off they are from each other. If the two are similar or approximately the same the plot should be a linear plot with a slope of ~1.   
\subsection{Hypothesis}
\label{sec:Hypothesis}
Before any computation we can already tell that method 2 (\ref{EQ2}) will be a more accurate conclusion than method 1 (\ref{EQ1}) since method 1 assumes all the mass is concentrated at a single point. This is an incorrect assumption as Dark matter halos can stretch out far beyond the disk and bulge as seen in Fig\ref{fig:fig 2}. The expectation is that the plot of method one will resemble a function of the form:
\begin{equation}
    y(x) = \frac{C}{\sqrt{x}}
    \label{EQ4}
\end{equation}
Where C is some constant that will be determined during computation. Method 2 will be more accurate since it will take into consideration the change in enclosed mass as we move further out into the halo. 
The expectation of the plot of \ref{EQ1}vs\ref{EQ2} is that the slope will either not be linear or the slope will not be ~1. The trend of the plotted line will show the discrepancies of the simplification approach.


%%%%%%%%%%%%%%%%%%%% REFERENCES %%%%%%%%%%%%%%%%%%

% The best way to enter references is to use BibTeX:

\bibliographystyle{mnras}
\bibliography{example} % if your bibtex file is called example.bib


% Alternatively you could enter them by hand, like this:
% This method is tedious and prone to error if you have lots of references
%\begin{thebibliography}{99}
%\bibitem[\protect\citeauthoryear{Author}{2012}]{Author2012}
%Author A.~N., 2013, Journal of Improbable Astronomy, 1, 1
%\bibitem[\protect\citeauthoryear{Others}{2013}]{Others2013}
%Others S., 2012, Journal of Interesting Stuff, 17, 198
%\end{thebibliography}


% Don't change these lines
\bsp	% typesetting comment
\label{lastpage}
\end{document}

% End of mnras_template.tex
